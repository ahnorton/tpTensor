
% Size ordering is...
% \big
% \Big
% \bigg
% \Bigg

\newcommand{\ED}{2{\small ED}}  % use as {\ED} in text to aviod blank-space problems

\newcommand{\LW}{Li\'enard--Wiechert}  % use as {\LW} in text to aviod blank-space problems
\newcommand{\FS}{Frenet--Serret}       % use as {\FS} in text to aviod blank-space problems
\newcommand{\be}{\begin{equation}}
\newcommand{\ee}{\end{equation}}

% These are in TeXit preamble:
% \newcommand{\bea}{\begin{eqnarray}}
% \newcommand{\eea}{\end{eqnarray}}

\newcommand{\diag}{{\rm diag}}
\newcommand{\ddd}{\dddot}          % just a name change
\newcommand{\dddd}{\ddddot}        % just a name change
\newcommand{\ndot}[2]{{\stackrel{{}_{(#1)}}{#2}}}

% The macros that use \tensor still take _ and ^  to define index position.
% eg.,  \kap{_1}  \kap{_1^2}

\newcommand{\N}[1]{\tensor{N}{#1}}
\newcommand{\kap}[1]{\tensor*{\kappa}{#1}}
\newcommand{\W}[1]{\tensor{W}{#1}}
\newcommand{\Y}[1]{\tensor{Y}{#1}}
\newcommand{\Yd}[1]{\tensor{\dot Y}{#1}}
% \newcommand{\e}[1]{\tensor{{\bf e}}{#1}}

\newcommand{\nN}{\tensor{n}{_N}} 
\newcommand{\dotnN}{\tensor{\dot n}{_N}} 
\newcommand{\thetaN}{\tensor{\hat\theta}{_N}}
\newcommand{\phiN}{\tensor{\hat\phi}{_N}}

\newcommand{\nY}{\tensor{n}{_Y}}
\newcommand{\dotnY}{\tensor{\dot n}{_Y}} 
\newcommand{\thetaY}{\tensor{\hat\theta}{_Y}}
\newcommand{\phiY}{\tensor{\hat\phi}{_Y}}

% Electromagnetic constants...
\newcommand{\muzero}{\tensor{\mu}{_0}}
\newcommand{\epszero}{\tensor{\epsilon}{_0}}


\newcommand{\mb}{\overline m}                     % bare mass from the MP equations 
\newcommand{\mB}{m_{{}_{\!B\!}}}             % bare mass 
\newcommand{\qB}{q_{{}_{\!B\!}}}             % bare charge 
\newcommand{\lamB}{\lambda_{{}_{\!B\!}}}     % bare wavelength
\newcommand{\me}{m_{\rm e}}                  % electron mass


\newcommand{\ret}{{\rm ret}}
\newcommand{\adv}{{\rm adv}}
\newcommand{\src}{{\rm src}}
\newcommand{\tot}{{\rm tot}}
\newcommand{\ext}{{\rm ext}}
\newcommand{\chg}{{\rm chg}}
% \newcommand{\chg}{{(z)}}
\newcommand{\sys}{{\rm sys}}
\newcommand{\inn}{{\rm in}}
\newcommand{\out}{{\rm out}}
\newcommand{\sph}{{\rm sph}}
\newcommand{\Max}{{\rm Maxwell}}
\newcommand{\mech}{{\rm mech}}
% \newcommand{\dip}{{\rm dipole}}
\newcommand{\dip}{{\rm dp}}
\newcommand{\edp}{{\rm e.dp}}
\newcommand{\mdp}{{\rm m.dp}}
\newcommand{\far}{{\rm far}}

\newcommand{\advret}{{\!\!\!\! \scriptsize {\begin{tabular}{l}\vspace{-11pt} \\ \adv \vspace{-3pt} \\ \ret \end{tabular}} \!\!\!\!\! }}

\newcommand{\free}{{\rm free}}
\newcommand{\ind}{{\rm ind}}
\newcommand{\rad}{{\rm rad}}
\newcommand{\inter}{{\rm int}}

\newcommand{\z}[1]{\tensor{z}{#1}}
\newcommand{\zd}[1]{\tensor{\dot z}{#1}}
\newcommand{\zdd}[1]{\tensor{\ddot z}{#1}}
\newcommand{\zddd}[1]{\tensor{\dddot z}{#1}}
\newcommand{\zdddd}[1]{\tensor{\ndot{4}{z}}{#1}}


% Haven't made up my mind what to use for the sphere radius, so use a macro...
\newcommand{\sr}{\tensor{r}{_\!_\circ_\!}}        %  old
\newcommand{\ro}{{\tensor{r}{_\!_\circ}}}         %  as used in Mathematica
\newcommand{\rod}{{\tensor{\dot r}{_\!_\circ}}}   
\newcommand{\rodd}{{\tensor{\ddot r}{_\!_\circ}}}       
\newcommand{\roddd}{{\tensor{\dddot r}{_\!_\circ}}}     
\newcommand{\rosqr}{{\tensor{r}{_\!_\circ^\!^2}}} %  ro^2 with the 2 nicely placed.
\newcommand{\rb}{{\bar r}}                        %  as used in Mathematica
% \newcommand{\rx}{{\tensor{r}{_x}}}                %  used in static potential
% \newcommand{\sx}{{\tensor{s}{_x}}}                %  used in static potential
\newcommand{\rx}{{r_x}}                %  used in static potential
\newcommand{\sx}{{s_x}}                %  used in static potential

% Tension...
\newcommand{\tauo}{\tensor{\Sigma}{_\circ}}   %  old

% Helix radius...
\newcommand{\rh}{{r_{\rm h}}}                    %  as used in Mathematica
% \newcommand{\rh}{{\tensor{r}{_\!_{\rm h}_\!}}}   %  as used in Mathematica

% A general reduced wavelength: lambar = lambda/(2pi)
\newcommand{\lambar}{\lambda\!\!\!\!\,\bar{\phantom{o}}}

% The codata standard for Compton wavelength, with a subscript C for Compton...
\newcommand{\lamC}{\lambda_{{}_{\!C}}}

% The codata standard for reduced Compton wavelength, with a subscript C for Compton...
\newcommand{\lambarC}{\lambar_{{}_{C}}}

% Always subscript, so no _ needed:   \an{n} == <k , z_n>  ...
\newcommand{\an}[1]{\tensor{a}{_(_#1_)}}

% Lagrangian, Euler operator, Noether current...
\newcommand{\Lag}{\mathscr{L}}
\newcommand{\LagO}{\tensor{\Lag}{_0}}                             % O not 0  (re. name)   Use \tLag instead.
\newcommand{\Lagl}{\tensor{\Lag}{_1}}                             % l not 1  (re. name)   Use \tLag instead.
\newcommand{\tLag}[1]{\tensor{\Lag}{#1}}                          % Usage: \tLag{_0}, \tLag{_1},...  tensor Lag.
% \newcommand{\Lagn}[2]{\tensor*{\Lag}{^{(#1)\,}_{\; #2}}}        % Usage: \Lagn{n}{\alpha}  n=0,1,2 
\newcommand{\emN}[1]{\tensor{\mathcal{N}}{_#1_{\!}}}              % Usage: \emN{v} 
\newcommand{\Dn}[2]{\tensor*{D}{^{(#1)\!}_{#2}}}                  % Usage: \Dn{n}{\alpha}  n=0,1,2 
\newcommand{\Lagn}[2]{\tensor*{D}{^{(#1)\!}_{#2}}\Lag}            % Usage: \Lagn{n}{\alpha}  n=0,1,2 
% Old defn...
\newcommand{\Euler}[1]{\tensor{\mathcal{E}}{_#1}}                 % Usage: \Euler{\alpha} 
% New defn...
\newcommand{\Eu}[1]{\tensor{\mathcal{E}}{#1}}                     % Usage: \Eu{_\alpha} \Eu{_i_j}  etc.   


% Labeled Lagrangians...
\newcommand{\Lmech}{\tensor{\Lag}{_{\rm mech}}}
\newcommand{\Lchg}{\tensor{\Lag}{_{\rm chg}}}

% The fundamental lengths. 
% Can not use numerals in command names!...
\newcommand{\lam}[1]{\tensor*{\lambda}{#1}}                       % usage  \lam{_1}, \lam{_1^2} 

% Lie deriv...  
\newcommand{\Lie}[1]{\tensor{\mathcal{L}}{_#1}}                   % Usage: \Lie{v} 

% Fast and slow spinors...
\newcommand{\psis}{\tensor{\psi}{_{\rm s}}}
\newcommand{\psif}{\tensor{\psi}{_{\rm f}}}
\newcommand{\dpsis}{\tensor{\dot \psi}{_{\rm s}}}
\newcommand{\dpsif}{\tensor{\dot \psi}{_{\rm f}}}

% 3-vector bold notation...
% \newcommand{\bbeta}{{\boldsymbol \beta}}
% \newcommand{\btheta}{{\boldsymbol \theta}}

% Two argument variation... 
\newcommand{\dvw}[2]{\tensor{\delta}{_{(#1,\,#2)}}}              % Usage: delta version:        \dvw{v}{\omega}
\newcommand{\Dvw}[2]{\tensor{\nabla}{_{\!(#1,\,#2)}}}            % Usage: nabla version:        \Dvw{v}{\omega}
\newcommand{\tDvw}[2]{\tensor{\tilde\nabla}{_{\!(#1,\,#2)}}}     % Usage: tilda nabla version:  \tDvw{v}{\omega}

% ---------------------------------------------------------------------------
% To keep cut and pastes working...

% Use this one if sub-scripted...
% eg.,  \ndotd{4}{z}{\alpha}
\newcommand{\ndotd}[3]{{\stackrel{{}_{(#1)}}{#2}}_{\!#3}}

\newcommand{\fsc}{{\alpha}}

% Schrodinger wave function...
\newcommand{\psiS}{\psi_{{\!}_S}}

\newcommand{\Sf}{|S|_{{}_{0}}}     % free-particle spin magnitude
\newcommand{\mf}{m_{{}_{0}}}       % free-particle mass

% Get rid of this...
\newcommand{\Gam}[2]{{\Gamma^{#1}_{\phantom{#1}#2}}}

\newcommand{\rH}{r_{{}_{\!H}}}
\newcommand{\wH}{\omega_{{}_{\!H}}}
\newcommand{\gH}{\gamma_{{}_{\!H}}}
\newcommand{\bH}{\beta_{{}_{\!H}}}
\newcommand{\aH}{\alpha_{{}_{\!H}}}
\newcommand{\mH}{m_{{}_{\!H}}}
\newcommand{\SH}{S_{{}_{\!H}}}

% -------------------------------------------------------------------------------
% From  http://tex.stackexchange.com/questions/96479/how-can-i-type-lambda-bar

% For a lower bar, 
% \makeatletter
% \newcommand{\lambdabar}{{\mathchoice
%   {\smash@bar\textfont\displaystyle{0.25}{1.2}\lambda}
%   {\smash@bar\textfont\textstyle{0.25}{1.2}\lambda}
%   {\smash@bar\scriptfont\scriptstyle{0.25}{1.2}\lambda}
%   {\smash@bar\scriptscriptfont\scriptscriptstyle{0.25}{1.2}\lambda}
% }}
% \newcommand{\smash@bar}[4]{%
%   \smash{\rlap{\raisebox{-#3\fontdimen5#10}{$\m@th#2\mkern#4mu\mathchar'26$}}}%
% }
% \makeatother

% For a bar that matches hbar, 
\newcommand{\lambdabar}{{\mkern0.75mu\mathchar '26\mkern -9.75mu\lambda}}

% -------------------------------------------------------------------------------

